%%% Surplus questions

%%% 2004 Exam, Q8,
%%% 2006 Exam, Q9,

\question[5]
Prove by mathematical induction that $5^{2n} + 17 \times 7^n$ is divisible by $18$, for all positive integers $n$.

\begin{EnvFullwidth}
\begin{solutionorgrid}[6in]
\begin{proof}
$P(n)$ is: $18 \divides 5^{2n} + 17 \times 7^n$, for all $n \in \Z^+$.

\textbf{Basis}: If $n = 1$, the LHS is $5^{2 \times 1} + 17 \times 7^1 = 25 + 17 \times 7 = 144$ and $18 \divides 144$. So, $P(1)$ is true.

\textbf{Inductive step}: Suppose $P(k)$ is true. Then $5^{2k} + 17 \times 7^k = 18m$, for all $k \in \Z^+$ and some $m \in \Z$. Thus, $5^{2k} = 18m - 17 \times 7^k$. Now,
\begin{align*}
	5^{2(k + 1)} + 17 \times 7^{k + 1} &= 5^2 \times 5^{2k} + 17 \times (7 \times 7^k) \\
	&= 5^2 \times (18m - 17 \times 7^k) + 7 \times (17 \times 7^k) && (\textrm{by hypothesis}) \\
	&= 25 \times 18m - 17 \times 7^k(25 - 7) \\
	&= 18(25m - 17 \times 7^k), 
\end{align*}
which is divisible by $18$. Since $P(1)$ is true and $P(k) \implies P(k + 1)$, by the PMI $P(n)$ is true for all $n \in \Z^+$.
\end{proof}
\end{solutionorgrid}
\end{EnvFullwidth}

\threeast

\uplevel{
In the following three theorems, $X$ denotes the domain of a real-valued function and $x \in X$ is a point in the domain. The notation $f: X \to \R$ means ``$f$ is a function whose domain is $X$ and whose range is $\R$.''

\begin{theorem}[Derivative of the identity function]
\label{thm:derivative-of-the-identity-function}
Let $X$ be a subset of $\R$ and let $f: X \to \R$ be the identity function defined as $f(x) = x$. Then $f'(x) = 1$.
\end{theorem}

\begin{theorem}[Product rule]
\label{thm:product-rule}
Let $X$ be a subset of $\R$, and let $f: X \to \R$ and $g: X \to \R$ be functions. Then $(fg)'(x) = f'(x)g(x) + f(x)g'(x)$.
\end{theorem}

\begin{theorem}[Power rule]
\label{thm:power-rule}
Let $n$ be a positive integer, let $X$ be a subset of $\R$ and let $f: X \to \R$ be the function defined as $f(x) = x^n$. Then $f'(x) = nx^{n - 1}$.
\end{theorem}
}

\question[5]
Prove Theorem~\ref{thm:power-rule} using mathematical induction. (\emph{Hint}: Use Theorems~\ref{thm:derivative-of-the-identity-function}~and~\ref{thm:product-rule}.)

\begin{EnvFullwidth}
\begin{solutionorgrid}[4in]
\begin{proof}
$P(n)$ is: if $f(x) = x^n$, then $f'(x) = nx^{n - 1}$, for all $n \in \N_{> 0}$.

\textbf{Basis}: If $n = 1$, then $f(x) = x^1 = x$ and $f'(x) = 1 \times x^{1 - 1} = x^0 = 1$. So, $P(1)$ is true.

\textbf{Inductive step}: If $P(k)$ is true, then $f(x) = x^k$ and $f'(x) = kx^{k - 1}$ for all $k \geq 1$. Now, if $f(x) = x^{k + 1} = x^k \times x^1$, by the product rule
\begin{align*}
	f'(x) &= \left( \diff{}{x} x^k \right) \times x + x^k \times \diff{}{x} x && (\textrm{by Theorem~\ref{thm:product-rule}}) \\
	&= kx^{k - 1} \times x + x^k \times 1 && (\textrm{by hypothesis and Theorem~\ref{thm:derivative-of-the-identity-function}}) \\
	&= kx^k + x^k \\
	&= (k + 1)x^k.
\end{align*}
Since $P(1)$ is true and $P(k) \implies P(k + 1)$, by the PMI if $f(x) = x^n$, then $f'(x) = nx^{n - 1}$, for all $n \in \Z^+$.
\end{proof}
\end{solutionorgrid}
\end{EnvFullwidth}

\triast

\uplevel{
\begin{theorem}[Binomial theorem]
\label{thm:binomial-theorem}
Let $x$ and $y$ be real numbers, let $n$ be a positive integer and let $\binom{n}{j}$ be a binomial coefficient. Then
\[
	(x + y)^n = \sum_{j = 0}^n \binom{n}{j} x^{n - j} y^j.
\]
\end{theorem}

\begin{theorem}[Pascal's rule]
\label{thm:pascal's-rule}
Let $n$ and $j$ be integers and let $\binom{n}{j}$ be a binomial coefficient. Then
\[
	\binom{n}{j - 1} + \binom{n}{j} = \binom{n + 1}{j}.
\]
\end{theorem}
}

\question[5]
Prove Theorem~\ref{thm:binomial-theorem} by mathematical induction. (\emph{Hint}: use Theorem~\ref{thm:pascal's-rule}.)

\begin{EnvFullwidth}
\begin{solutionorgrid}[5.5in]
\begin{proof}
$P(n)$ is: For all $n \in \Z^+$
\[
	(x + y)^n = \sum_{j = 0}^n \binom{n}{j} x^{n - j} y^j.
\]

\textbf{Basis}: If $n = 1$, then the LHS is $(x + y)^1 = x + y$ and the RHS is
\[
	\binom{1}{0} x^{1 - 0} y^0 + \binom{1}{1} x^{1 - 1} y^1  = x + y.
\]
So, $P(1)$ is true.

\textbf{Inductive step}: If $P(k)$ is true, then for all integers $k \geq 1$
\[
	(x + y)^k = \sum_{j = 0}^k \binom{k}{j} x^{k - j} y^j.
\]
Now,
\begin{align*}
	(x + y)^{k + 1} &= (x + y)^k (x + y) \\
	&= \left( \sum_{j = 0}^k \binom{k}{j} x^{k - j} y^j \right) (x + y) \\
	\begin{split}
	&= x \left[ \binom{k}{0} x^k y^0 + \binom{k}{1} x^{k - 1} y^1 + \binom{k}{2} x^{k - 2} y^2 + \cdots + \binom{k}{k} x^0 y^k \right] \\
	&+ y \left[ \binom{k}{0} x^k y^0 + \binom{k}{1} x^{k - 1} y^1 + \binom{k}{2} x^{k - 2} y^2 + \cdots + \binom{k}{k} x^0 y^k \right] \\
	\end{split} \\
	\begin{split}
	&= \left[ \binom{k}{0} x^{k + 1} y^0 + \binom{k}{1} x^k y^1 + \binom{k}{2} x^{k - 1} y^2 + \cdots + \binom{k}{k} x^1 y^k \right] \\
	&+ \left[ \binom{k}{0} x^k y^1 + \binom{k}{1} x^{k - 1} y^2 + \binom{k}{2} x^{k - 2} y^3 + \cdots + \binom{k}{k} x^0 y^{k + 1} \right] \\
	\end{split} \\
	&= \binom{k}{0} x^{k + 1} y^0 + \left[ \binom{k}{0} + \binom{k}{1} \right] x^k y^1 + \left[ \binom{k}{1} + \binom{k}{2} \right] x^{k - 1} y^2 + \cdots + \binom{k}{k} x^0 y^{k + 1} \\
	&= \binom{k + 1}{0} x^{k + 1} + \binom{k + 1}{1} x^{(k + 1) - 1} y + \binom{k + 1}{2} x^{(k + 1) - 2} y^2 + \cdots + \binom{k + 1}{k + 1} y^{k + 1} \\
	&= \sum_{j = 0}^{k + 1} \binom{k + 1}{j} x^{(k + 1) - j} y^j.
\end{align*}
Since $P(1)$ is true and $P(k) \implies P(k + 1)$, by the PMI $P(n)$ is true for all $n \in \Z^+$.
\end{proof}
\end{solutionorgrid}
\end{EnvFullwidth}
